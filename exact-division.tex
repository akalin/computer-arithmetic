\documentclass{computer-arithmetic}

\usepackage{computer-arithmetic}
\usepackage{algorithm}
\usepackage{algpseudocode}

\begin{document}

\section{Notes on Exact Division}

The problem of exact division is to, given a non-negative integer
\(a\) and a positive integer \(b \mid a\), compute the integer \(q\)
such that \(a = qb\).

Any division algorithm can compute this, but computing the remainder
is wasted work, so maybe we can do something faster using the
\(b \mid a\). Note that a division algorithm that returns only the
quotient gets us most of the way there, but ``remainder is 0'' is a
stronger precondition than ``remainder is unused'', so there's still
room for improvement.

The first key insight is that we can reduce this problem to the
problem of modular inversion. Since \(a = qb\), for any modulus \(n\),
\[
  a ≡ qb \pmod n\text{,}
\]
and if \(b\) has a modular inverse \(b'\) mod~\(n\), then
\[
  q ≡ ab' \pmod n\text{.}
\]
If \(n\) is large enough, which we can guarantee since we know the
size of \(a\) and \(b\), then \(q \bmod n = q\).

The assumption that \(b\) has an inverse mod~\(n\) isn't a serious
obstacle. \(b\) has an inverse mod~\(n\) iff \(\gcd(b, n) =
1\). Therefore, if \(b\) doesn't have an inverse mod~\(n\), then let
\(m = \gcd(b, n)\), which is greater than \(1\). But since
\(b \mid a\), \(m\) must also be a factor of \(a\), and we can divide
both \(a\) and \(b\) by \(m\). If \(n = β^k\) and \(β\) is a prime
power (e.g., \(β = 2^l\)), then \(m\) must also be a power of the
prime base, and division by \(m\) reduces to shifting. At worst, we'd
have to do two short divisions, which might be better than a long
division.

Now then what should we pick for \(n\)? A natural choice would be a
power of \(β\). In particular, if \(b\) has size \(n'\) and \(a\) has
size \(m + n'\), then we know that \(q\) has size at most \(m + 1\),
so we can set \(n = β^{m+1}\).

\subsection{Computing inverses mod \(β^k\) and Jebelean's exact division algorithm}

So now the problem is reduced to, given \(b\) such that
\(\gcd(b, β^k) = 1\), compute \(b'\) such that \(bb' ≡ 1 \pmod
β^k\). We'll see in later sections, but the most straightforward
algorithm is the extended GCD algorithm. However, the first step of
that would be to compute \(\operatorname{divmod}(β^k, b)\), which
doesn't seem to be that much better than computing \(a/b\)!

The next insight is to start with finding the lowest word of the
quotient. Using the example from
\url{https://gmplib.org/manual/Exact-Division}, let \(A = 368154\) and
\(B = 543\). We want to find \(Q\) such that \(A = QB\). Now reduce
everything mod~\(10\), and we get \(A ≡ QB \pmod{10}\), which is
\[
  4 ≡ q_0 3 \pmod{10}\text{,}
\]
where \(q_0\) is the lowest digit. Then multiply both sides by \(7\) the
inverse of \(3\) \(\pmod{10}\) to get
\[
  q_0 ≡ 4 ⋅ 7 ≡ 8 \pmod{10}\text{.}
\]
Then subtract \(8B\) from \(A\) to get
\[
  368154 - 8 ⋅ 543 = 368154 - 4344 = 353810\text{,}
\]
Note that the result now has \(0\) as the lowest digit. Now we can
repeat the same process with the next digit until we have the full
quotient. This leads us to
\href{https://www.sciencedirect.com/science/article/pii/S0747717183710126}{Jebelean's
  exact division algorithm}:

\begin{algorithm}
  \caption{JebeleanExactDivision: Calculate \(Q\) such that
    \(A = Q ⋅ B\), where \(b\) is a length-\(n\) slice containing the
    digits of the positive integer \(B\), \(a\) is a
    length-\((n + m)\) (\(m ≥ 0\)) slice containing the digits of the
    non-negative integer \(A\), \(B \mid A\), and
    \(\gcd(b_0, β) = 1\). Returns the array \(q\) containing the
    digits of \(Q\).}
  \begin{algorithmic}[1]
    \State \(q ← \operatorname{NewArray}()\)
    \State \(r ← a.\operatorname{copy}()\)
    \State \(b' ← \operatorname{modInv}(b_0, β)\)
    \While {\(R > 0\)}
    \State \(t ← (r_0 ⋅ b') \bmod β\)
    \State \(q.\operatorname{push}(t)\)
    \State \(R \mathrel{-}= B ⋅ t\)
    \Comment Lowest word of \(R\) should now be \(0\)
    \State \(R.\operatorname{rightShiftWords}(1)\)
    \EndWhile
    \State return \(q\)
\end{algorithmic}
\end{algorithm}

\subsection{What does exact division do if the precondition is violated?}

Remove the \(b \mid a\), and let
\((q, r) = \operatorname{divmod}(a, b)\), so that \(a = qb + r\). Then
for any modulus \(n\),
\[
  a ≡ qb + r \pmod n\text{,}
\]
and if \(b\) has a modular inverse \(b'\) mod~\(n\), which only
happens if \(\gcd(b, n) = 1\), then
\[
  q ≡ ab' - rb' \pmod n\text{,}
\]
so the quotient that we get will be off by an additive factor of
\(-rb'\) (mod~\(n\)).

\end{document}
