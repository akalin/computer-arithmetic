\documentclass{computer-arithmetic}

\usepackage{computer-arithmetic}
\usepackage{algorithm}
\usepackage{algpseudocode}

\begin{document}

\section{Notes on Algorithm~1.6 and Algorithm~1.8}

To derive Algorithm~1.8 (RecursiveDivRem), we basically do something
similar as what we did with Karatsuba multiplication. For Karatsuba
multiplication, we
\begin{enumerate}
\item took as given a function for single-word multiplication, which
  takes in two words and returns their double-word product,
\item wrote double-word multiplication, which takes in two
  double-words and returns their quadruple-word product, in terms of
  single-word multiplication,
\item optimized the sequence of operations, and
\item replaced the calls to single-word multiplication with recursive
  calls, added a base case, and replaced 'words' with 'halves'.
\end{enumerate}

For recursive div-rem, we will
\begin{enumerate}
\item take as given a function for double-word division, which takes
  in a double-word dividend, a single-word divisor, and returns their
  double-word quotient and single-word remainder,
\item write quadruple-word division, which takes in a quadruple-word
  dividend, a double-word divisor, and returns their triple-word
  quotient and double-word remainder, in terms of double-word
  division,
\item optimize the sequence of operations, and
\item replace the calls to double-word division with recursive calls,
  add a base case, and replace 'words' with 'halves'.
\end{enumerate}

We start by rewriting Algorithm~1.6 like so:

\begin{algorithm}
  \caption{BasecaseDivRem: Calculate \(Q\) and \(R\) such that
    \(A = Q ⋅ B + R\) and \(0 ≤ R < B\), where \(b\) is a length-\(n\)
    slice containing the digits of the positive normalized integer
    \(B\), and \(a\) is a length-\((n + m)\) (\(m ≥ 0\)) slice
    containing the digits of the non-negative integer \(A\). The
    length-\((m+1)\) array \(q\) containing the digits of \(Q\) and
    the length-\(n\) array \(r\) containing the digits of \(R\) will
    be output.}
  \begin{algorithmic}[1]
    \State \(q ← \operatorname{NewArray}(\operatorname{fill}=0, \operatorname{length}=m + 1)\)
\State \(r ← a.\operatorname{copy}()\)
\If {\(A ≥ B.\operatorname{leftShiftWords}(m)\)}
\State \(q_m ← 1\)
\State \(R \mathrel{-}= B.\operatorname{leftShiftWords}(m)\)
\Else
\State \(q_m ← 0\)
\EndIf
\For {\(j\) from \(m-1\) down to \(0\)}
\State \(q_j ← \operatorname{ClampedDoubleWordDiv}([r_{n+j}, r_{n + j - 1}], b_{n-1})\)
\Comment Returns the max word value on overflow
\State \(R \mathrel{-}= B.\operatorname{leftShiftWords}(j).\operatorname{mulWord}(q_j)\)
\While {\(R < 0\)}
\State \(q_j \mathrel{-}= 1\)
\State \(R \mathrel{+}= B.\operatorname{leftShiftWords}(j)\)
\EndWhile
\EndFor
\State return \((q, r)\)
\end{algorithmic}
\end{algorithm}

The clamped double-word division is necessary in order to guarantee a
single-word output. However, if we didn't clamp, the while loop in
lines 12-14 would still guarantee that \(q_j\) fit in a single word,
(at the expense of performance guarantees, but we won't worry about
that yet). If we're also willing to let the first iteration produce a
double-word, we could get rid of the initial if-statement:

\begin{algorithm}
  \caption{BasecaseDivRem v2}
  \begin{algorithmic}[1]
    \State \(Q ← 0\)
\State \(r ← a.\operatorname{copy}()\)
\For {\(j\) from \(m-1\) down to \(0\)}
\State \(t ← \operatorname{DoubleWordDiv}([r_{n+j}, r_{n + j - 1}], b_{n-1})\)
\State \(R \mathrel{-}= B.\operatorname{leftShiftWords}(j).\operatorname{mulDoubleWord}(t)\)
\While {\(R < 0\)}
\State \(t \mathrel{-}= 1\)
\State \(R \mathrel{+}= B.\operatorname{leftShiftWords}(j)\)
\EndWhile
\State \(Q = Q.\operatorname{leftShiftWords}(1) + t\)
\EndFor
\State return \((Q, r)\)
\end{algorithmic}
\end{algorithm}

Now we can specialize this to quadruple-word division by unrolling the loop.

\begin{algorithm}
  \caption{QuadrupleDivRem: Calculate \(Q\) and \(R\) such that
    \(A = Q ⋅ B + R\) and \(0 ≤ R < B\), where \(b\) is a length-\(2\)
    slice containing the digits of the positive normalized integer
    \(B\), and \(a\) is a length-\(4\) slice
    containing the digits of the non-negative integer \(A\). The
    length-\(3\) array \(q\) containing the digits of \(Q\) and
    the length-\(2\) array \(r\) containing the digits of \(R\) will
    be output.}
  \begin{algorithmic}[1]
\State \(r ← a.\operatorname{copy}()\)

\State \(Q_1 ← \operatorname{DoubleWordDiv}([r_3, r_2], b_1)\)
\State \(R \mathrel{-}= B.\operatorname{leftShiftWords}(1).\operatorname{mulDoubleWord}(Q_1)\)
\While {\(R < 0\)}
\State \(Q_1 \mathrel{-}= 1\)
\State \(R \mathrel{+}= B.\operatorname{leftShiftWords}(1)\)
\EndWhile

\State \(Q_0 ← \operatorname{DoubleWordDiv}([r_2, r_1], b_1)\)
\State \(R \mathrel{-}= B.\operatorname{mulDoubleWord}(Q_0)\)
\While {\(R < 0\)}
\State \(Q_0 \mathrel{-}= 1\)
\State \(R \mathrel{+}= B\)
\EndWhile

\State return \((Q_1.\operatorname{leftShiftWords}(1) + Q_0, r)\)
\end{algorithmic}
\end{algorithm}

\end{document}
