\documentclass{computer-arithmetic}

\usepackage{computer-arithmetic}
\usepackage{algorithm}
\usepackage{algpseudocode}

\begin{document}

\section{Notes on Algorithm~1.3}

\begin{itemize}
\item The input line
  \[
    A = \sum_0^{n-1} a_i β^i \qquad \text{and} \qquad
    B = \sum_0^{n-1} b_i β^i
  \]
  is defining two sets of things:
  \begin{itemize}
  \item the integers \(A\) and \(B\), and
  \item their respective unique base-\(β\) digits, which would
    conventionally be written
    \[
      a_{n-1} a_{n-2} \dotso a_1 a_0 \qquad \text{and} \qquad
      b_{n-1} b_{n-2} \dotso b_1 b_0\text{,}
    \]
    with the given equations relating the two. A computer program
    would normally represent two inputs in array-like structures of
    length~\(n\), where e.g. the \(i\)th element of the array
    \(\mathbf{A}\) would hold \(a_i\).
  \end{itemize}
\item The output line
  \[
    C = ∑_0^{2n-1} c_k β^k
  \]
  is similarly defining the integer \(C\) and its unique base-\(β\)
  representation
  \[
    c_{2n-1} c_{2n-2} \dotso c_1 c_0\text{.}
  \]
  Note that it's implicitly saying that \(C\) can have at most \(2n\)
  base-\(β\) digits.
\item The line
  \[
    (A_0, B_0) := (A, B) \bmod β^k, (A_1, B_1) := (A, B) \operatorname{div} β^k
  \]
  is perhaps better written as:
  \begin{align*}
    A_1 &:= a_{n-1} \dotso a_k \\
    B_1 &:= b_{n-1} \dotso b_k \\
    A_0 &:= a_{k-1} \dotso a_0 \\
    B_0 &:= b_{k-1} \dotso b_0\text{,}
  \end{align*}
  i.e. there's no actual division or modulo operation being done.
\item It's unclear to me what the difference between an assignment (\(←\))
  and a definition (\(:=\)) is.
\item It's probably better to name \(C_2\) as \(t\) and \(C_1\) as
  \(C_2\) to be consistent with Toom-Cook.
\end{itemize}

I'd rewrite it as

\begin{algorithm}
  \caption{KaratsubaMultiply: Calculate \(C = A × B\), where \(a\) and
    \(b\) are length-\(n\) slices containing the digits of \(A\) and
    \(B\) respectively, and the length-\(2n\) array \(c\) containing
    the digits of \(C\) will be output.}
\begin{algorithmic}[1]
\If {\(n < n_0\)}
    \State return $\operatorname{BasecaseMultiply}(a, b)$
\EndIf
\State $k ← \lceil n / 2 \rceil$
\State \((a_1, a_0) ← (a.\operatorname{slice}(k, n), a.\operatorname{slice}(0, k))\)
\State \((b_1, b_0) ← (b.\operatorname{slice}(k, n), b.\operatorname{slice}(0, k))\)
\State \((s_a, d_a) ← a_0 - a_1\)
\Comment Destructure the subtraction result into its sign and magnitude.
\State \((s_b, d_b) ← b_0 - b_1\)
\State \(c_0 ← \operatorname{KaratsubaMultiply}(a_0, b_0)\)
\State \(c_2 ← \operatorname{KaratsubaMultiply}(a_1, b_1)\)
\State \(t_0 ← c_0 + c_2\)
\State \(t_1 ← \operatorname{KaratsubaMultiply}(d_a, d_b)\)
\If {\(s_a s_b > 0\)}
\State \(t_0 \mathrel{-}= t_1\)
\Comment guaranteed to be non-negative.
\Else
\State \(t_0 \mathrel{+}= t_1\)
\EndIf
\State \(c ← \operatorname{NewArray}(\operatorname{fill}=0, \operatorname{length}=2n)\)
\State \(c \mathrel{+}= c_0\)
\State \(c.\operatorname{slice}(k, 2n) \mathrel{+}= t_0\)
\State \(c.\operatorname{slice}(2k, 2n) \mathrel{+}= c_2\)
\State return \(c\)
\end{algorithmic}
\end{algorithm}

\begin{problem}{8-9}
\end{problem}

\begin{solution}
\end{solution}

\end{document}
